\chapter{مقدمه}
\label{ch:fasl1}
\pagenumbering{arabic} %برای اینکه شماره‌گذاری از حالت حرفی به عددی تبدیل شود.

\قسمت{انگیزه}
در  علوم مهندسی، شبیه‌سازی باعث کاهش هزینه‌ها و نوآوری در طراحی می‌شود. یکی از علوم مهندسی که دانش شبیه‌سازی 
به پیشرفت آن کمک می‌کند، جریان چند‌فازی در محیط متخلل است. این علم در صنایع متعددی کاربرد دارد، از جمله: مدیریت آب‌های
 زیرزمینی، مهندسی مخزن و مهندسی ژئوتکنیک. برای مثال در زمینهٔ آب‌های زیرزمینی انتخاب یک روش مناسب برای دفع زباله‌های 
 شیمیایی به طوری که کمترین آسیب را به این منابع بزند، نیاز به شبیه‌سازی راهکارهای گوناگون دارد\مرجع{basthabil}. 
 در مهندسی ژئوتکنیک، با توجه به اینکه رفتار مکانیکی خاک با تغییر میزان اشباع\پانویس{Saturation} فاز‌های مختلف در آن تغییر می‌کند، 
 حل توزیع سیالات در خاک در شبیه‌سازی رفتار مکانیک سازه اهمیت پیدا می‌کند. امّا کاربردی که در این رساله مورد توجه ماست، 
 شبیه‌سازی جریان سیالات در مخازن هیدروکربنی است. چرا که مدیریت مناسب منابع هیدروکربنی در گرو پیش‌بینی مناسب توانایی تولید و 
 انتخاب روش بهینه برای افزایش بازدهی منابع می‌باشد و یکی از روش‌های دست‌یابی به این مهم، شبیه‌سازی فرآیند استخراج نفت و 
 جریان آن در مخزن می‌باشد.

با توجه به اینکه بخش عظیمی از درآمد کشورمان از صنعت نفت تأمین می‌شود، تحقیق و توسعه در این زمینه اهمیت زیادی پیدا می‌کند. این صنعت به دو بخش صنایع بالادستی و صنایع پایین‌دستی تقسیم می‌شود. مهندسی مخزن و شبیه‌سازی مخازن نفتی جزئی از صنایع بالادستی محسوب می‌شود. در این حیطه مخازن نفتی به دو دسته معمولی\پانویس{Conventional} و نامعمول\پانویس{Unconventional} تقسیم می‌شوند. مخازن معمولی، رفتار قابل پیش‌بینی‌تری نسبت به مخازن نامعمول دارند. این مخازن فاقد ترک\پانویس{Fracture} هستند و سنگ تشکیل‌دهنده آن‌ها از نظر خواص مهندسی نسبتاً همگن\پانویس{Homogenous} و همسان\پانویس{Isotropic} است. شبیه‌سازی مخازن نامعمول به دلیل وجود ترک‌ها، ناهمسانی و ناهمگنی شدید و \نقاط‌خ{} فرآیند پیچیده‌تری می‌باشد. یکی از مثال‌های مخازن نامعمول، مخازن ترک‌دار طبیعی هستند. در این مخازن به دلیل حرکت‌های زمین‌ساختی،  ترک‌های متعددی (از مقیاس‌های خیلی کوچک تا مقیاس‌های بزرگ) به وجود آمده‌اند. دستهٔ مهم دیگری از مخازن نامعمول را مخازن شیل\پانویس{Shale} تشکیل می‌دهند. در این مخازن برای برداشت نفت و گاز باید به طور مصنوعی در اطراف چاه‌ها ترک ایجاد شود. در هر دو حالت ذکر شده مدل‌سازی ترک‌ها و تأثیر آن‌ها در فرآیند‌های افزایش برداشت، از اهمیت ویژه‌ای برخوردار است.

 مخازن ترک‌دار طبیعی بخش قابل توجهی از مخازن نفتی خاورمیانه را تشکیل می‌دهند\مرجع{sarma}. دقت در بررسی رفتار این مخازن می‌تواند به سود اقتصادی عظیمی منجر شود در حالی که عدم شناخت مناسب می‌تواند ضرر‌های هنگفتی را به بار آورد. برای مثال، در سال ۱۹۷۸ در میدان بیور‌ریور\پانویس{Beaver River} در کانادا، تولید کم و غیر قابل پیش‌بینیِ بزرگترین میدان گازی ایالت بریتیش کلمبیا باعث تعجب بسیاری شد\مرجع{vangolf} بر خلاف آن، در سال ۱۹۵۱ در میدان نفتی مارالاپاز در ونزوئلا که دارای تخلخل ماتریس کمتر از 3 درصد و نفوذپذیری در حدود یک دهم میلی دارسی بود، تولید روزانه ۲۵۰٬۰۰۰ بشکه صورت می‌گرفت\مرجع{vangolf}. اهمیت عظیم مدلسازی مخازن ترک‌دار طبیعی مشوق اصلی این پروژه می‌باشد.

\قسمت{هدف}
روش‌های متعددی برای مدل‌سازی ترک‌ها وجود دارند که بسیاری از آن‌ها در \مرجع{berkowitz} عنوان شده‌اند. دو خانواده روش‌های متدوال روش‌های ترک
گسسته\پانویس{Discrete Fracture} و روش‌های ترک پیوسته\پانویس{Continuum Fracture} می‌باشند. در روش‌های ترک پیوسته فرض می‌شود که محیط ترک و محیط ماتریس هر دو محیط‌هایی پیوسته هستند که در محیط متخلخل قرار گرفته‌اند و جریان سیال در هر دو محیط وجود دارد، به علاوه فرض می‌شود که این دو محیط سیال را با یکدیگر نیز مبادله می‌کنند. در روش ترک گسسته مدل کردن ناهمسانی و ناهمگنی خواص در محیط متخلخل اهمیت خیلی بیشتری نسبت به حالت قبلی دارد. زمانی که این ناهمگنی و ناهمسانی مدل شود ترک چیزی جز قسمتی از محیط متخلخل با خواصی متفاوت نخواهد بود. معمولا با فرض اینکه متغیر‌ها در ضخامت ترک تغییر نمی‌کنند (به دلیل ضخامت خیلی کم نسبت به ماتریس) می‌توان روش‌های عددی را پایدار‌تر و سریعتر کرد. با توجه به این مطلب اهداف این پروژه به شرح زیر هستند.
\شروع{tight_enumerate}
\فقره ساخت یک برنامه محاسباتی برای شبیه‌سازی جریان دوفاز غیرامتزاجی\پانویس{Immiscible} و تراکم ناپذیر، با ویژگی‌های:
	\شروع{tight_itemize}
	\فقره حل مسائل در هندسه‌های یک و دو‌بعدی.
	\فقره استفاده از یک روش حجم محدود\پانویس{Cell Centered Finite Volume} بر روی شبکه محاسباتی بدون سازمان\پانویس{Unstructured Grid}.
	\فقره استفاده از روش \متن‌لاتین{IMPES} برای خطی سازی معادلات.
	\فقره مدل‌سازی ناهمگنی و ناهمسانی خواص سنگ و خواص سیالات.
	\فقره قابلیت مدل‌کردن ترک‌ها با استفاده از المان‌های دوبعدی\پانویس{Fine Mesh Solvers} و المان‌های یک یعدی
	\فقره مدل کردن شرایط مرزی مختلف.
	\پایان{tight_itemize}
\فقره بررسی صحت\پانویس{Verification} برنامه تولید شده.
\پایان{tight_enumerate}
درا ین پروژه سعی شده است که ويژگی‌های برنامه ساخته‌شده و مدل سیالاتی استفاده شده در عین سادگی پرکاربرد نیز باشند.

\قسمت{مروری بر کارهای انجام شده}
برنامه‌ای که باید در این پروژه ساخته شود، توسط محققان متعددی به همین روش، روش‌های مشابه و البته روش‌های متفاوتی ساخته شده‌است. در این قسمت تعدادی از این مقالات را که در انجام این پروژه از آن‌ها کمک گرفته‌ایم و یا اینکه ارتباط نزدیکی با این پروژه دارند، نام خواهیم برد.

دو مقاله \مرجع{baca} و \مرجع{noorishad} از اوّلین تلاش‌ها برای پیاده‌سازی روش ترک گسسته به شمار می‌روند. در این مقالات جریان سیال تک‌فاز به هم‌راه انتقال حرارت و انتقال محلول در محیط متخلخل ترک‌دار، به روش گالرکین\پانویس{Upwinded Galerkin} مدل‌سازی شده است. کریمی‌فرد و فیروز‌آبادی\مرجع{karimi1} همین روش را برای شبیه‌سازی جریان دوفازی غیرامتزاجی تعمیم دادند. روش گالرکین ساده به دلیل \متن‌لاتین{conservative} نبودن، نمی‌تواند ناهمگنی و ناهمسانی شدید خواص سیال و سنگ را به درستی مدل کند و نسبت به روش‌هایی که در ادامه بیان خواهیم کرد از مقاومت\پانویس{Robustness} کمتری برخوردار است.

بستیان در \مرجع{bastn,basthabil} از یک روش حجم محدود گره مرجع\پانویس{Vertex Centered} و فرمولاسیون‌های متعدد  برای شبیه‌سازی  جریان دوفازی در محیط متخلخل با ناهمگنی بالا استفاده کرد. گروه تحقیقاتی همین محقق متعاقباً امکان استفاده از المان‌های با بعد کمتر برای مدل‌سازی ترک‌ها را نیز به روش خود اضافه کردند\مرجع{reichpaper,reichthes}. در تمامی این مقالات از روش‌های کاملاً ضمنی\پانویس{Fully Implicit} برای خطی‌سازی  معادلات استفاده شده‌است. مونتیگودو و فیروزآبادی\مرجع{mont1,mont2} روشی مشابه با همان قابلیت‌ها را پیاده‌سازی کردند. آن‌ها بر خلاف گروه قبلی از روش  \متن‌لاتین{IMPES} برای خطی سازی معادلات استفاده کردند. آن‌ها در ادامه از روش کاملاً ضمنی با فرمولاسیون متفاوتی نسبت به گروه اوّل برای خطی سازی معادلات استفاده کردند  و این دو روش را با یکدیگر مقایسه نمودند\مرجع{montimpj,montimpc}. روش استفاده شده در این پروژه نیز مشابه روش‌های دو گروه نام‌برده  می‌باشد. در طول این پروژه دو اشکال قابل توجه در این روش یافت شده‌اند. اوّل این که روش ما قادر نیست که ترک‌هایی که جلوی جبهه جریان را می‌گیرند مدل کند. اشکال دیگر این  است که در این روش با اینکه دو متغیر اشباع و فشار با دقت خوبی محاسبه می‌شوند، متغیرهای شار دقت کمتری دارند. با وجود این کاهش دقت، پاسخ‌های حاصل برای اهداف پروژهٔ کنونی کفایت می‌کنند.

کریمی‌فرد و عزیز \مرجع{karimi2}، روش حجم محدود سلول مرجعی\پانویس{Cell Centered} را برای روش ترک گسسته طراحی کردند که از روش‌های گره مرجع
 ذکر شده ساده‌تر بود. به علاوه امکان مدل کردن ترک‌هایی که مانع جریان هستند را داشت. اشکال این روش این بود که از یک سلول محاسباتی  دو نقطه‌ای\پانویس{Two Point Flux Approximation Scheme - TPFA} برای محاسبه شار استفاده می‌کرد و این امر باعث می‌شد که این روش نتواند ناهمسانگردی تراوایی مطلق\پانویس{Absolute Permeability} را مدل کند و فقط برای ماتریس‌های همسانگرد مناسب باشد. آواتسمارک و همکاران و ادواردز و همکاران روش‌های حجم محدود خاص خود را برای شبیه‌سازی محیط بدون ترک طراحی کرده‌اند\مرجع{aavatsmark,edwards1}. این روش‌ها اخیراً برای مدل‌کردن ترک‌های گسسته تعمیم داده شده‌اند. ادعا شده است که این روش جدید هر دو مشکل عنوان شده برای روش ما را حل خواهند کرد\مرجع{edwards2}.

دسته‌ی دیگری از روش‌ها که برای شبیه‌سازی جریان چند‌فازی در محیط متخلخل به کار می‌روند و دقت بالایی در محاسبه شار دارند، روش‌های ترکیبی هستند. در این روش‌ها که معمولاً به کمک روش \متن‌لاتین{IMPES} خطی می‌شوند، معادله فشار به کمک روش المان محدود ترکیبی\پانویس{Mixed Finite Element} و معادله اشباع به کمک یک روش \متن‌لاتین{conservative} مثل حجم محدود\مرجع{helmig} یا گالرکین گسسته\پانویس{Discrete Galerkin}\مرجع{hoteitn} حل می‌شوند. این روش‌ها نیز برای مدل‌سازی ترک‌های گسسته تعمیم داده شده‌اند\مرجع{hoteitf}. تنها عیبی که می‌توان به این روش‌ها گرفت این است که نسبت به روش‌های قبلی بسیار پیچیده‌تر هستند و پیاده‌سازی آن‌ها نیازمند تجربه بالا در زمینه روش‌های المان محدود است.

\قسمت{ساختار این رساله}

مراحل زیر در این رساله دنبال خواهند شد:
\شروع{tight_itemize}
\فقره ابتدا در فصل \رجوع{ch:fasl2} خواص مهم سیالات و سنگ‌ها از جمله فشار مویینگی بیان خواهند شد. سپس معادلات حاکم بر سیستم و شرایط مرزی معرفی می‌شوند.
\فقره در فصل \رجوع{ch:fasl3} روش عددی برای حل معادلات در دو بعد معرفی خواهد شد.
\فقره در فصل \رجوع{ch:fasl4} مسائل متعددی از مراجع معتبر برای بررسی صحت برنامه حل خواهند شد.
\پایان{tight_itemize}
