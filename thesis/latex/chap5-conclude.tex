\chapter{نتیجه گیری و پیشنهادها} 
\label{ch:fasl5}
در این رساله یک روش حجم محدود برای شبیه‌سازی جریان دوفازی در محیط متخلخل ترک‌دار به روش ترک گسسته بیان شد. با اندکی کار جبری معادلات حاکم بر مسئله به یک معادله بیضوی برای فشار (پتانسیل آب) و یک معادله هذلولوی برای اشباع آب تبدیل گشتند. سپس این معادلات به یک روش حجم محدود گره مرجع \text‌لاتین{CVFEM} گسسته‌سازی شدند. از روش \lr{IMPES} نیز برای دیکوپل کردن دستگاه معادلات استفاده شد. روش \text‌لاتین{CVFEM} می‌توانست بدون تدابیر اضافی ناهمگنی و ناهمسانی تراوایی مطلق را مدل کند، امّا برای مدل کردن ناهمگنی فشار مویینگی تدابیر اضافی را اتخاذ نمودیم. در آخر برای بررسی صحت کد محاسباتی سه مسئله شاخص را حل نمودیم. دو مسئله اوّل به بررسی ناهمگنی فشار مویینگی و جاذبه می‌پرداختند و در مسئله آخر مدل کردن ترک‌ها بررسی گشت. 

کارهای ذیل جهت ادامه تحقیق حاضر پیشنهاد می‌شوند:
\begin{tight_itemize}
\item روش ترک گسسته قادر است که وجود ترک‌ها را با دقت بالایی مدل کند، امّا با زیاد شدن تعداد ترک‌ها زمان محاسباتی مورد نیاز آن بسیار زیاد می‌شود. لذا در پروژه‌ها صنعتی از روش‌های دیگری که به نام روش‌های ترک پیوسته شناخته می‌شوند، استفاده می‌شود. در این روش‌ها پارامترهایی به نام توابع انتقال وجود دارند که معمولاً تقریبی از آن‌ها به صورت تجربی و یا تحلیلی محاسبه می‌شود. روش ترک گسسته می‌تواند راه جدیدی را برای محاسبه توابع انتقال ارائه دهد. لذا پیشنهاد می‌شود که این دو روش با یکدیگر مقایسه شوند و سعی شود که از روش ترک گسسته جهت ارتقا روش‌های ترک پیوسته استفاده شود.
\item مدل جریان دوفاز غیرامتزاجی برای شروع تحقیق در زمینه روش ترک گسسته مناسب می‌باشد. امّا برای حل مسائل واقعی  و پر کاربرد در صنعت نیاز به تعمیم روش ترک گسسته برای حل مدل‌های جریان پیشرفته‌تر مثل نفت سیاه احساس می‌شود. لذا پیشنهاد می‌شود که روش پروژه حاضر برای حل معادلات نفت‌سیاه تعمیم داده شود.
\item در تحقیق حاضر ناهمگنی فشار مویینگی به کمک حل تحلیلی تعدادی معادله مرتبط با فشار مویینگی در بخش \ref{ch:352} مدل شد. به دلیل استفاده از حل تحلیلی محدودیتی در کار وجود دارد که تمام نواحی باید دارای یک منحنی $J(S)$ باشند. در صورتی که بتوانیم معادلات حاضر را به روش عددی حل نماییم این محدودیت برطرف می‌شود. به علاوه می‌توانیم به جای توابع ریاضی از درون‌یابی نقاط گسسته برای محاسبه مقادیر $J(S)$ استفاده نماییم. این کار شاید در مدل جریان دوفاز غیرامتزاجی اهمیت زیادی نداشته باشد امّا در مدل‌های پیشرفته‌تر مثل نفت سیاه الزامی است.
\item روش \text‌لاتین{CVFEM} محدودیت‌هایی دارد. مهمترین آن‌ها این است که سرعت فاز‌ها را با دقت خیلی کمتری نسبت به متغیر‌های اشباع و فشار حل می‌نماید. در صورت نیاز به دقت بالاتر برای محاسبه شار‌ها می‌توان به سراغ روش حجم محدود ارائه شده در \cite{edwards2} و یا روش  \text‌لاتین{MFE}\cite{hoteitf,hoteitn} برای حل معادله فشار رفت. از بین این دو پیاده‌سازی روش اوّل ساده‌تر می‌باشد و پیشنهاد می‌شود که از آن استفاده شود. محدودیت دیگر عدم امکان مدل کردن ترک‌هایی است که جلوی جریان را می‌گیرند. استفاده از روش‌های سلول مرجع به جای گره مرجع\cite{edwards2,karimi2} می‌تواند این مشکل را برطرف سازد.
\item استفاده از روش کاملاً ضمنی به جای روش \lr{IMPES} نیز می‌تواند زمان محاسباتی را کاهش دهد و راندمان کد محاسباتی را افزایش دهد.
\end{tight_itemize}
